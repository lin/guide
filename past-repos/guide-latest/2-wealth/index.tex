\chapter{章二:醒目的美味}

这一节,主要让人知道,能够完成基本的生存也是很困难的。

可以说说历史,说说进化,说说分工

其实这一章的内容是非常重要的,因为他涉及到了大多数人,大多数时间精力的问题

每个人的生活的主体,都在这个方面

\section{呼吸的代价}

艰苦的工作挣钱,和放纵的享受生活,对于历史上的大多数人,都是矛盾的,但前者只是途径,后者才是目的。

如果你享受生活的方式可以帮你赚钱,便是一种幸运。

\subsection{为什么总有太多的不情愿和苦差事?}
\subsection{为什么一定要工作?挣钱和花钱的本质是什么?}
\subsection{为什么工作的喜爱程度不一样,活的质量就不一样?}
\subsection{为什么钱是一生中最重要的概念之一?}

\section{贴身的枷锁}

如果你现在有两个亿的存款,你还愿意继续做你正在做的事情吗?

如果没有了别人的眼光和看法,没有了外界的承认,没有了虚幻的来世回报,你还会继续同样的生命挥霍吗?

这一章,主要是强调一个基本的生活标准的达到也是要付出代价的

\subsection{为什么说大多数人活在被奴役的状态?}
\subsection{积累财富的方式和路径有哪些?}
\subsection{为什么给别人打工往往不是一个最佳选择?}
\subsection{如何设计更好的产品、提供更好的服务?}

被恐惧所奴役,被财务所奴役,被虚荣所奴役,被集体性虚荣所奴役,被传统所奴役,被情感所奴役

我之所以讨论家庭和情感少,是因为只不过是把所有的问题乘以了一个小于十的常数。

\section{迟钝的味蕾}
大口吃肉,大口喝酒,困了睡,醒了玩。
\subsection{高中生心态:为什么拜金和势利的人是肤浅的?}

那些只认钱与权的人是可怕又可怜的,可怕在于对人性的蔑视,可怜在于对生活品味之浅。

\subsection{不在于你赚多少钱,而是你怎么赚的钱}
