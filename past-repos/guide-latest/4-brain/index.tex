\chapter{章四:木制的宝剑}
这一章的矛盾在于:大脑是很厉害的,但也有自己的局限性。

主要在这一章要表达的是:人脑的局限,以及简单且错误的模型

这一张和《thinking fast and slow》有很多重合之处,但是我想强调的是系统性

 简单的模型,低分辨率的理解和图像

 感性的去理解,缺乏充分的分析

 你可以写文章,但是如果你是为了出名和赚钱,你可能就要取宠,

1. 以偏概全(第一大失误),rush to conclusion,对分类的过分应用,粗糙的对信息的处理,不能细致的分析信息,以偏概全

c错误的抽象和总结归类

2. 相信权威,思维不爱动脑,喜欢走捷径,喜欢相信标签
3. 强加因果,张冠李戴,胡说瞎扯,缺乏逻辑,过分依赖对比(某些相同特征 推出 其他特性也一致)
4. 脸面大于正误,先说自己不能错,因为丢人,然后才是对问题的分析
5. 模型简单,线性模型,略微先进一点的是,二次模型,光谱模型,二元系统,三元系统,细致地分析
6. 局域化的思考方式,只关心当前局部的得失
7. 群体化思维严重

过分简化的模型,羸弱的模型系统

都有什么模型?基本的操作有哪些?

人们常用的模型是什么?为什么这个模型容易放错?

什么是高分辨率?什么是低分辨率?

人就是想节省计算力,少用大脑,这样可以节约能量。而热爱思考和学习的群体就是当时的bug群体。

常见的错误model:

1. 认为人生的成功就是财富和名望的成功。

\section{为什么所有人都有进化的痕迹?}
% \section{什么是人行动的源动力?}
\section{为什么人不爱思考?为什么人总是短视?}
此处要讲很多人思考问题的基本模型 局域 脑力消耗

\section{为什么不要盲目相信传统和标签?}
为什么传统智慧是大众化的?为什么传统智慧总是滞后?
% \section{为什么尚古和标签化思考是知识匮乏、盲从和虚荣的体现?}
% \section{为什么不要对传统和大众盲目随从?}
