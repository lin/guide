\section{死亡的恐怖}

\subsection{超越想象的重返生前}

即使,所有活着的人都未曾体验。但不难推断,死亡就是重返生前。

死后的感觉就是没有感觉。一切暗色无光,一切毫无乐趣,短暂到瞬间,却又漫长到无尽。

人活着,可以去理解,可以去体验,可以去创造,可以去挑战,人死了,一切可能的可能,都变成不可能,所有的信息都因失去载体而变得毫无意义。

现实世界中的一切幻象随之散落,任何的骄傲,任何的难堪,所有的期许与牵挂,所有可能的探索与渴求,一切的一切,都在一瞬间坍塌,一切的一切,都毫无回转得,陷入不尽的窒息和沉寂。

\subsection{如果可以活到一万岁?}

和仅仅两百年前相比,我们拥有远超帝王的舒适生活。汽车和飞机的出现,仅仅是最近一百年左右的事,流行乐的产生发生在五十年前,互联网的问世不过三十年,智能手机的时代也只是十年之前才开启,没有这些工具和作品,我们的人生又会局限到多小呢,又会枯燥多少呢?

如果时光可以倒流,如果可以回到十年前,大抵每个人都有后知后觉的感悟和重头再来时的不同选择,可惜人生太过短促,很多遗憾已经无法弥补。当你知道第一学历的重要性时,你已经不敢再去参加一次高考。当你知道婚姻的幸福不该建立在是恋爱时的冲动,你已经为人父母,只能接纳现实的无奈。

如果我们有更长的生命,哪怕活到二百岁,三十岁的你,也可以像十七岁的约翰$\cdot$列侬\footnote{John Lennon(1940 - 1980),英国披头士乐队(The Beatles)创始成员}那样去创作,五十岁时怀揣类似马斯克\footnote{Elon Musk(1971 - ), 美国太空探索技术公司(SpaceX)创始人}三十岁时的远志。幸运的不仅仅是可以找到热爱的事业,而且是可以在合适的年龄、理想的外界环境下去选择自己的一生所爱。

如果我们生命足够长,食物和其他生活必需品更加廉价充裕,如果每个人可以在三十岁才开始选择自己的努力方向后开始积累,而不是更多按部就班的盲从,被迫的去选择简单乏味的工作,如果每个人可以从五十岁开始工作,陪自己孩子到成年,这样的人生要比现在的日子要幸福多少呢?

想象一百年后,随着星际探索的进步和人工智能的发展,我们将拥有难以享尽的生活资源,身上没有了生存的枷锁,再也没有了枯燥单调的工作类型,每个人都可以自由的去发挥想象,尽情的探索和创造。这些都仅仅是多活了一百年而已,没有人能够想象生命长度是一万岁的复杂和精彩。

但令人吃惊的是,并不是所有人都那么珍惜自己宝贵的生命,一个极端的例子,那些为了德国纳粹或者日本军国主义效命的军兵,为了某种虚无缥缈的情结和难以兑现的回报,他们过早的结束生命,如果他们知道自己要活到一万岁,还会选择一条高概率死在二十几岁的道路吗?

生命实在过于宝贵去尝试这些低级的愚蠢,活着还有改错的机会,为了短期的冲动和利益,而无悔坠入深渊,万一你错了呢?所得的不过是生前短暂的满足,失去的却是无尽可能的生命。

\subsection{如果原本可以活到一万岁?}

生活只是逻辑可能中极度微小的一段,死亡让人失去的是无穷大,而不是仅仅少了两位数。

那些生命跨度平均不足百年的先人束缚了我们对人生的认识,先人的常态似乎成了无限可能中的唯一可能,人生成了一连串的游园打卡,本来无穷尽的逻辑可能,变成了屈指可数的体验项目,还要忍受难以避免的排队和不敢走出乐园的恐慌。迫不得已挤占着稀少的时光,本就局限的感知被落后的本能和传统固封在一个狭小的区域。短短几千年的历史没办法定义人生奋斗的意义,区区五亿平方公里的海陆也定义不了人生可以探索的边界。但现实如重力,它不在乎任何人的感受和得失,它只管亘古不变地把一切拉回地面。

人生的丰富,逻辑上所有可能的精彩,只可能在活着的时候去体验。人生有着超越想象的庞大,对此,我们最好做坏的假设,例如一个人至少要活一万岁才能无憾终止,而你仅仅活了百岁,就好像一个刚满岁的婴童因病早逝,尚未体会人生本可以带来的一切欣喜和可能,就匆匆离世。而我们这些活了不到百岁的人,若是安慰自己没白活一生,或是挥霍本就稀少的时光,便实在是令人惋惜。

一个不尽的悬崖,正渐渐向我们每个人逼近,无论我们想不想。一点一点,直到把我们拽回到那无穷尽的黑暗和枯燥之中,越是意识到死亡的恐怖和无穷,越是让人意识到生存的严肃和可贵,让人珍惜时间的稀缺和紧迫,让人通过死亡的视角,去看待那些悲哀又可笑的短视和局限。

$\sim$ 1800 words
