\section{生活的美好}

\subsection{给生活中的乐趣分类、分级}
我们并不是列举,衣食住行,所有的乐趣,而是为了更加方便我们分析人生的乐趣。

第一个层级:生存,如果从,我的爷爷和太爷作为生活在20世纪的中国农民,他们真的仅仅维持在了生存这一关。
第二个层级:基本的生活水准。这个略有tricky,和当时科技水平相关,也就是一般人的生活水准。

一方面体现的比例,中值。一方面体现的是科技和文明的进步水平。

当然这里没办法罗列所有可能的乐趣,更重要的是分类。分类的标准和意义。

物质上的有:食物(饮食的快感),居住环境,生活用具,子女的教育,医疗,旅游的乐趣,娱乐
精神上的有:和睦的家庭关系,朋友关系,喜爱的工作

第三个层级:进阶的生活水准。也就是同时代较为稀有的资源和享受,并不是每一个人都能够轻松得到的资源。

物质上,比如说,17世纪英国的菠萝。物质与财富
精神上,比如说,被人追捧和羡慕,有很强的优越感。超出众人的乐趣。权利和名望

这种优越,体现在享受生活的滞后性,如果一件事即使随着时代的发展也没办法普及给普通人,那么就不算是进阶的生活品质了。

对于以上三个层面的享受,有着线性的关系,也就是往往越多越好。而前三个层面有着一定的可以被左右的奴役色彩。

高中生心态,对世界的复杂度缺乏认识的一种体现,认为只有一种可能,不知道其他的可能性。世界的复杂度就是,只要你仔细的去思考,很多事情都比你自己

第四个层级:更为丰富和更加个性化的生活体验。
第五个层级:带有个人兴趣爱好的追求。
第六个层级:挑战和创造。

Challenging yourself, stepping out the comfortable zone, and aiming a high goal will give you more insterests and understand broader in life, and always you will say how I wish I understand this earlier.

从第四个层面开始,出现了选择的自由和多元化的世界。有了自己可以选择的专业方向,可以学习更为深入的而非常识性的知识,可以为自己的长久理想做真正的准备。

可以用上学的知识来类比各个层级,第一个层级是小学知识,基本的语言交流和算术。第二个层级是初中,学会了一些初级的技巧。第三个层级是高中,上了很多的难度,但大家学习的内容基本还是一致的,不过开始出现了严重分化的现象,只有很小一个比例的人可以在高中结束时完成大的突破,进入一个理想的大学。

大家似乎不会迷恋初中太久,就算你可以再一个圆的问题上做出十几条的辅助线,也不可能证明你的教育水平有多好,毕竟到了高中才会系统学习牛顿力学和进化论。

但是就算你在高考的时候分数在93\%以上(高考约700分以上),你也不过是一个高中生,对世界的理解和体验非常的有限,甚至连微积分这样的基础概念都不太熟悉。

\subsection{为什么维持生存异常艰辛?}
\subsection{高档生活的乐趣有哪些?}
\subsection{活出自由指的是什么?}
\subsection{为什么只有获得自由之后才是真正的活着?}
\subsection{如何活出精彩?为什么喜好、挑战和创造是更高级的乐趣?}
\subsection{为什么达到自由之后,选择就难有优劣之分?}
