\section{生活的美好}

上一节,本文讨论了死亡的沉寂。这一节,我们将对活着的乐趣尽可能的明确量化,从而把人生问题转化为数学上的优化问题。

\subsection{给生活中的需求分级}

这里的讨论有些类似于马斯洛的需求层级理论,但本文的侧重点更在于对需求进行量化,从而将有大量假设前提,这是全书的基础。

为了使得读者更加明确地这种分级的概念和必要性,本文将采用类比的方式进行辅助解释。

\subsubsection*{第一个层级:生存级别}

这一级别的基本特点就是保证能过维持人在肉体上的稳定存在。其他野生动物的生活的需求也大多数停留在这个层面。例如,

\begin{itemize}
    \item 保证食物的基本供应,防止饥渴造成的死亡
    \item 保证不被其他人类成员或集体剥夺生命,也就是谋杀
    \item 保证威胁生命的重大疾病或重大意外发生的几率足够低
\end{itemize}

\subsubsection*{第二个层级:同时代基本生活标准}

这个级别的需求包含了大多数人在日常生活中的大多数需求,也就是日常中所说的普通老百姓的生活。例如,

\begin{itemize}
    \item 能够得到基本的教育,掌握常识
    \item 能够拥有一个稳定的住所
    \item 能够找到配偶,组建一个家庭
    \item 能够为子女提供足够的成长必需品以及基本的教育
    \item 能够支付基本医疗上的支出
    \item 能够购买生活常用品,例如,家具,厨具,服饰,文具图书等
    \item 能够获得身边人基本的尊敬和认可
\end{itemize}

这个层级的需求最大的特点是被大多人所共享,也就是大多数人在这个层级的需求是相近无差的。

这个层级的需求是随着时代的进步而变化的。例如,在十七世纪英国,菠萝是财富和地位的象征,属于同时代高档次生活标准,而在二十一世纪,我们把可以购买菠萝视为同时代的基本生活标准。

这个层级的需求对于不同地区的人对基本标准的定义也有所不同,例如,当下发达国家和发展中国家的生活标准是有很大差距的。而本文中定义,应该是每个人在可以预期或视野范围内的生活标准,既可以是身边人的平均标准,也甚至是最发达地区所享有的生活标准,这与个体的预期有关。

\clearpage

\subsubsection*{第三个层级:同时代高档次生活标准}

与第二个层级相似,这个层级的大多数需求同样被大多人所共享,与此同时,这个层级的需求不可避免的只能被少数人所达到,这种不可避免一般是由于资源的有限和竞争造成的。例如,

\begin{itemize}
    \item 能够得到同时代的最好教育,例如,中国的985和美国的常青藤盟校,由于资源有限,注定只能有少数人通过竞争达到。
    \item 能够拥有可以产生优越感的知名标签,例如,上述的名校学习,在一个知名的企业工作甚至拥有高级职位,与知名人士或组织有交集
    \item 能够拥有大量的财富,购买高价商品和服务,甚至可以不必投入时间精力而得到持续的现金流,也就是财富自由,因为价值的创造需要大量人的投入参与,所以拥有财富自由只能是少数人可以达到。
    \item 能够获得名望,也就是成为名人,为人所敬仰和钦慕。
\end{itemize}

\subsubsection*{第四个层级:个人偏好和潜能发展}

这个层级的需求相比之前三个层级,主要体现在多元化,也就是每个人的需求不尽相同,例如,

\begin{itemize}
    \item 艺术上的偏好,音乐,书法,电影,建筑,游戏等
    \item 学科探索上的偏好,考古,物理学,计算机科学,生物学,心理学等
    \item 改变生活上的偏好,环保,交通,产品设计,等
\end{itemize}

在这个层级上,本文有以下三个重要的假设:

\begin{enumerate}
    \item 个人偏好上的需求满足比高档次生活享受更加强烈和持久,也就是高档次生活的享受相对更加空洞和短暂
    \item 不同偏好上的需求满足很难有高低的判断,也就是并不能说对于音乐的偏好优越于考古
    \item 个人偏好无法既多又精,也就是全才的出现。尤其是在二十一世纪的背景下,各个领域被充分探索和尝试,内化一个常见偏好需要大量的时间和精力投入
\end{enumerate}

资源不稀缺,需求不共享,一般来说需求的满足也不昂贵,付出的代价不大。

\subsubsection*{第五个层级:挑战和创造}

这个层级是第四个层级的延伸,相当于高档次的个人偏好。由于不具备资源的稀缺性,每个人都可以选择挑战和创造。这个级别需求的满足,强调更多的是在挑战和创造过程中得到的满足,而不是挑战成功后得到的成就满足。

\begin{enumerate}
    \item 艺术上的创作
    \item 体育运动上的自我挑战
    \item 具有创新的创业活动
    \item 超出个体当前期望的挑战和尝试
\end{enumerate}

2020.11.11

\subsection{需求层级的一些规律}

\begin{enumerate}
    \item 第一个层级是乘法关系
    \item 后几个层级具有不同的order
\end{enumerate}

$$f(x_0, x_1, x_2, x_3) = \left(\frac{1}{1 + e^{-\lambda_0 x_0}} - \frac{1}{2}\right) \left( \lambda_1 x_1 + \lambda_2(x_2 + 1)\log (x_2 +1) + \lambda_3 x_3^2 \right)$$

\subsection{需求层级和学校教育层级的类比}