\chapter{章六:来世的回报}

这一章,主要讲人类社会的基本观念是如何形成的。

\section{经济是如何运行的?社会是如何进步的?}
局域的稀缺性和局域的需求强烈程度,解决问题的效率是如何提升的?
\section{如何给人分类?什么是社会阶层?}

以及这种传统观念的滞后性,以及其他看似快但是有害的思维方式
\section{为什么统治者既是贪婪的强盗又是公道的仆人?}
\section{如何评价文明的进展?为什么读史更应该关注小人物?}
普通人的生活质量和效率得到了提高
之前历史的错误性,在于对那些伟人的描述,而不是对普通人的描述,重要的,大多数人都是普通人,都是劳苦的大众,之所以我们在乎应该从普通人的角度出发

社会对死亡的认识,对时间的珍惜
\section{文明是如何从农耕进步到工业时代的?从工业进步到智能时代的?}
资源越少,拥有权利的优势就越大。但也只是一个层面的改善。只有大量的资源充裕,无需用生命去争取权利,而是去享受生活才是真正的优势。
