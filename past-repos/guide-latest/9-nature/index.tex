\chapter{章九:必要的偏执}

这一章比较好写,都是我比较擅长的领域

就是把知识体系用比较简单的话解释一遍,再说一下迷信,宗教和伪科学

\section{为什么迷信是愚昧?宗教是自我安慰?哲学是空中楼阁?}
Religion

假设,你想从一个流媒体软件上下载一首《hey jude》

首先,你的手机必须精准的识别你的手指以及手指的位置

接下里,你的手机必须对你的每一操作,进行准确的反馈,例如你敲击了h,屏幕上的键盘h要高亮,搜索栏的h要被画出来,

当你点击搜索的时候(甚至还没等点击搜索,就已经有自动提示了),你的信息要通过互联网访问流媒体的服务器,在流媒体的服务器上,同样需要搜索计算,把结果再发给手机,手机得到信息后,把信号转号称音频播放出来。

几乎每一秒,都有几十亿的基本逻辑操作要进行,而且基本逻辑在多个层面上得到了验证,不仅仅是写这个软件的人程序在执行的时候是准确,也说明了手机的操作系统的执行也没有出错,同时为了实现逻辑操作的晶体管在量子力学层面的假设也是靠得住的。

在一首歌的时间,近百亿次的验证着科学的严谨和可靠。

而宗教,在自己所描述的所有神迹事件,没有任何一件可以重现,是什么原因让宗教给人了无限的信任呢?

\section{什么是科学?为什么要崇尚科学?各学科之间的关系是什么?}
\section{什么是数学?为什么数学如此确定和无处不在?}
\section{什么是逻辑?为什么计算机科学如此的强大和万能?}
\section{愚公的智慧}
\section{什么是工程学?为什么要像狐狸一样解决问题?}
\section{什么是理性思考和批判性思维?为什么要理性思考?}
\section{什么是存在的原因?什么又是存在的价值?}
\section{为什么再聪明的人也有上限?什么是问题的复杂度?}

% The essential knowledge
% \section{什么是一切逻辑的源头和根基?}
% 什么是可区分性?
% \section{为什么计算机科学如此的强大和万能?}
% 什么是明确的计算?什么是通用图灵机?
% \section{什么是抽象?什么是数学?}
% 什么是抽象?什么是抽象层级?什么是计算的等效性?
% 发明与发现,复数,负数
% \section{为什么再聪明的人也有上限?什么是问题的复杂度?}
% \section{什么是概率?为什么概率统计是生活中最重要的概念之一?}
% \section{什么是归纳?为什么科学是如此的靠谱和值得信赖?}
% 什么是归纳和相关性?什么是可能近似正确?
% \section{什么是推演?为什么可以用模型推测过去和预测未来?}
% 什么是推理和因果性?如何综合归纳和推理?
% \section{什么是自然科学?科学的本质是什么?}
% \section{为什么有那么多学科?各学科之间的关系是什么?}
% 什么是科学的层级?
% \section{为什么迷信是愚昧?宗教是自我安慰?哲学是空中楼阁?}
% \section{什么是工程学?为什么要像狐狸一样解决问题?}
% \section{什么是理性思考和批判性思维?为什么要理性思考?}
